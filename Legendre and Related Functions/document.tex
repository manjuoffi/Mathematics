\documentclass[english,12pt,a4paper,titlepage]{scrbook}
\usepackage[T1]{fontenc}
\usepackage[left=2cm, right=2cm, top=2cm, bottom=2cm]{geometry}
\usepackage{graphicx}
\usepackage{mathtools}
\usepackage{amssymb}
\usepackage{amsthm}
\usepackage{thmtools}
\usepackage{xcolor}
\usepackage{nameref}
\usepackage{babel}
\usepackage{hyperref}
\usepackage{imakeidx}
\makeindex

\newtheorem{theorem}{Theorem}[section]
\newtheorem{remark}[theorem]{Remark}
\newtheorem{note}[theorem]{Note}
\newtheorem{recall}[theorem]{Recall}
\newtheorem{definition}[theorem]{Definition}
\newtheorem{exercise}[theorem]{Exercise}
\newtheorem{question}[theorem]{Question}
\newtheorem{lemma}[theorem]{Lemma}
\newtheorem{proposition}[theorem]{Proposition}
\newtheorem{example}[theorem]{Example}

\title{Legendre functions}
\subtitle{A collection of points from different sources}
\author{Manjunatha M R}
\begin{document}
	\maketitle %Creates tittle page
	\tableofcontents % Add a table of contents
	
	\part{Book: Higher transcendental functions, by Harry Bateman}
	\chapter{introduction}
	\section{}
	The expansion of $(a^2 - 2arcos(\gamma)+ r^2)^{-\frac{1}{2}}$ in powers of $r$ contains the coefficients in terms of $cos(\gamma$). These coefficients are polynomials of $cos(\gamma)$. These were first introduced in 1784 by Legendre and are known as Legendre polynomials. The expression $(a^2- 2arcos(\gamma)+ r^2)$ is dealing the potential at a point $P$ of a source at the point $A$. Here $r$ and $a$ are the distances of $P$ and $A$ from $O$ (Origin) respectively.  
	\chapter{Solution of Legendre Differential equations}
	\section{Legendre functions and differential equations}
	\begin{definition}
		[Legendre's differential equation] A differential equation of the form
		\[(1-z^2)\frac{d^2 \omega}{dz^2} -2z\frac{dw}{dz}+ [\nu(\nu+1) - \frac{\mu^2}{(1-z)^2}]\omega=0,\]
		where $\nu$,$\mu$ and z are unrestricted(Complex numbers)
	\end{definition}
	\begin{definition}[Legendre functions]\label{Legendre functions}\index{Legendre functions}
		The Legendre functions are solutions of Legendre's differential equation.
	\end{definition}
	\part{Book: Asymptotic and special functions, by F.W.J Olver}\label{part:two}
	\printindex
\end{document}